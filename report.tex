\documentclass{article}

\usepackage{graphicx} % Required for inserting images

\usepackage{amsmath} % AMS mathematical facilities for LATEX
\usepackage{amsfonts} % TEX fonts from the American Mathematical Society
\usepackage{bbold} % A geometric sans serif blackboard bold font, for use in mathematics;

\usepackage{float} % Improved interface for floating objects

\usepackage{listings} % The package enables the user to typeset programs (programming code) within LATEX
\lstset{language=Python}
\lstset
{ %Formatting for code in Python
    basicstyle=\footnotesize,
    numbers=left,
    stepnumber=1,
    showstringspaces=false,
    tabsize=1,
    breaklines=true,
    breakatwhitespace=false,
}

\setlength{\parindent}{0pt}
\usepackage{geometry} % Flexible and complete interface to document dimensions
\usepackage{todonotes}
\geometry{hmargin=2.5cm,vmargin=2.5cm}

\title{EDIN01 Cryptography \\ Project 2}
\author{Maxime Pakula, Sofia Boselli Graf}

\begin{document}

\maketitle

\tableofcontents

\newpage


\section{Polynomials over finite fields}

\subsection{Home Exercise 1}
In this exercise the task is to determine if given polynomials are reducible, irreducible or primitive. 

\begin{enumerate}
    \item $p(x) = x^4 + x^2 + 1 \text{ over } \mathbb F_2$ \\
    As $\mathbb F_2$ only contains two elements (0,1) the amount of possible polynomials which can me multiplied to obtain the above expression is small. In this case, to obtain a polynomial of degree 4 the multiplication must be done with polynomials of degree 2 and 2 or 1 and 3. Starting with 2 and 2, there are only 4 polynomials in $\mathbb F_2$ with this constraint, and only 2 with an independent term. Calculating long division with these two yields that a reminder 0 is obtained with $x^2 + x + 1$ and:
    $$(x^2+x+1)(x^2-x+1) = x^4 + x^2 + 1$$
    which means the polynomial $p(x)$ is reducible.
    \item $p(x) = x^3 + x + 1 \text{ over } \mathbb F_3$ \\
    In this case $\mathbb F_3$ has three elements (0,1,2). Instead of trying every possible alternative, possible roots of $p(x)$ are checked. By doing this it can be seen that $p(0) = 1$ but $p(1) = 3 = 0$ which means that there exists a way to reduce $p(x)$ with $(x-1)$ which in $\mathbb F_3$ is $(x+2)$. Calculating the results with long division the following is obtained:
    $$(x+2)(x^2-2x+2) = x^3 - 2x + 4 = x^3 + x + 1$$
    \item  $p(x) = x^2 + \alpha^5 x + 1 \text{ over } \mathbb F_{2^4}$ \\
    The first step is to obtain the multiplication table for $\mathbb F_{2^4}$ in order to obtain every element in the group and its expression in terms of $\alpha$ (if constructed from a primitive). If $p(x)$ is irreducible it means that $p(x) = (x-\alpha^i)(x-\alpha^j) = x^2 + (\alpha^i + \alpha^j)x + \alpha^{i+j}$. This means that for $p(x)$ to be reducible $(\alpha^i + \alpha^j) = \alpha^5$ and  $\alpha^{i+j} = \alpha^15$. By trying the possible alternatives a result is found with $i = 9, j=6$ which means $p(x)$ is reducible.
\end{enumerate} 

\subsection{Laboratory Exercise 1}
Now a polynomial is deemed reducible or not in Maple as follows:
\begin{enumerate}
    \item $p(x) = x^23 + x^5 + 1 \text{ over } \mathbb F_2$ \\
    \begin{verbatim}
        Irreduc(x^23 + x^5 + 1) mod 2
    \end{verbatim}
    This returns: \textbf{True} $\rightarrow$ Irreducible
    To check if it is primitive:
    \begin{verbatim}
        Primitive(x^23 + x^5 + 1) mod 2
    \end{verbatim}
    This returns: \textbf{True} $\rightarrow$ Primitive
    \item $p(x) = x^23 + x^6 + 1 \text{ over } \mathbb F_2$ \\
        \begin{verbatim}
        Irreduc(x^23 + x^6 + 1) mod 2
    \end{verbatim}
    This returns: \textbf{False} $\rightarrow$ Reducible
    \item $p(x) = x^18 + x^3 + 1 \text{ over } \mathbb F_2$ \\
        \begin{verbatim}
        Irreduc(x^18 + x^3 + 1) mod 2
    \end{verbatim}
    This returns: \textbf{True} $\rightarrow$ Irreducible
    To check if it is primitive:
    \begin{verbatim}
        Primitive(x^18 + x^3 + 1) mod 2
    \end{verbatim}
    This returns: \textbf{False}
    \item $p(x) = x^8 + x^6 + 1 \text{ over } \mathbb F_7$ \\
        \begin{verbatim}
        Irreduc(x^8 + x^6 + 1) mod 7
    \end{verbatim}
    This returns: \textbf{False} $\rightarrow$ Reducible
    \item $p(x) = x^6 + \alpha^5 x + 1 \text{ over } \mathbb F_{2^4}$ \\
    \todo[inline]{Dont know how}
\end{enumerate}

\subsection{Home Exercise 2}
In this exercise the order of the following elements must be determined, this is the smallest $t$ such that $a^t = 1$. In this case from the $\mathbb F_{2^4}$ multiplicative table it is known that $\alpha^15 = 1$ which means every element below should be taken to this number.
\begin{enumerate}
    \item $\alpha \rightarrow \alpha^t = 1 \rightarrow t = 15$
    \item $\alpha^2 \rightarrow (\alpha^2)^t = 1 \rightarrow t = 15$
    \item $\alpha^3 \rightarrow (\alpha^3)^t = 1 \rightarrow t = 5$
    \item $\alpha^5 \rightarrow (\alpha^5)^t = 1 \rightarrow t = 3$
    
\end{enumerate}
\subsection{Laboratory Exercise 2}
In this section the same task as before is done in Maple. The finite field is defined as:
\begin{verbatim}
    G218 := GF(2, 18, alpha^18 + alpha^3 + 1)
\end{verbatim}
And $\alpha$ as:
\begin{verbatim}
    a := G218:-ConvertIn(alpha)
\end{verbatim}
After this, the order of each desired element is calculated below.
\begin{enumerate}
    \item $\alpha$ \\
    \begin{verbatim}
        a := G218(alpha)
        G218:-order(a)
    \end{verbatim}
    Which returns: \textbf{189}
    \item $\alpha^2$ \\
    \begin{verbatim}
        b := G218:-`^`(a, 2)
        G218:-order(b)
    \end{verbatim}
    Which returns: \textbf{189}
    \item $\alpha^3$ \\
    \begin{verbatim}
        c := G218:-`^`(a, 3)
        G218:-order(c)
    \end{verbatim}
    Which returns: \textbf{63}
    \item $\alpha^3$ \\
    \begin{verbatim}
        d := G218:-`+`(c, a)
        G218:-order(d)
    \end{verbatim}
    Which returns: \textbf{262143}
\end{enumerate}
\subsection{Home Exercise 3}
\subsection{Laboratory Exercise 3}
\subsection{Home Exercise 4}
\subsection{Laboratory Exercise 4}


\section{De Bruijn Sequences}

\subsection{Home Exercise 5}
\subsection{Laboratory Exercise 5}


\end{document}
